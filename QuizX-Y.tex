%=========================================================================================
% 「パターン認識」レポート用TeXフォーマット
%
%       必ず,このフォーマットを利用してください.
%
%       1行目 - 30行目:     【変更不可】
%      31行目 - 47行目:     (1)課題番号 (2)氏名・学籍番号 (3)提出年月日 以外は変更しない
%      48行目以降:          ご自由にどうぞ. 
%                           ただし,最終行の \end{document} を消さないように!
%=========================================================================================
\documentclass[12pt,a4j,dvipdfmx]{jarticle}

\usepackage[textwidth=45zw,lines=40]{geometry}  % Text format 指定
%\usepackage[top=2cm, bottom=2cm, left=1cm, right=1cm]{geometry}
\usepackage{ascmac}                 % ASCII macro
\usepackage{amsmath}                % 数式スタイル
\usepackage{nccmath}                % 数式スタイル
\usepackage{amssymb}                % 数学記号
\usepackage{amsthm}                 % 数学・定理・証明スタイル
\usepackage{multicol}               % 部分的段組み
\usepackage{tabularx}               % 表スタイル
\usepackage{graphics}               % 図挿入スタイル
\usepackage{graphicx}               % 図挿入スタイル
\usepackage{cases}                  % 場合分け
\usepackage[usenames]{color}        % カラーの文書
\usepackage[dvipsnames]{xcolor}     % カラーの文書
\usepackage{extarrows}              % 長い矢印
\usepackage{bm}                     % \bm Bold Math for Vector and Matrix
\usepackage{fancybox}               % 色々な箱
\usepackage{listings,jlisting}  %日本語のコメントアウトをする場合jlistingが必要
%\usepackage{listings}  %日本語のコメントアウトをする場合jlistingが必要
% 以下,ソースコード表示に関する設定
\lstset{
  basicstyle={\ttfamily},
  identifierstyle={\small},
  commentstyle={\smallitshape},
  keywordstyle={\small\bfseries},
  ndkeywordstyle={\small},
  stringstyle={\small\ttfamily},
  frame={tb},
  breaklines=true,
  columns=[l]{fullflexible},
  numbers=left,
  xrightmargin=0zw,
  xleftmargin=3zw,
  numberstyle={\scriptsize},
  stepnumber=1,
  numbersep=1zw,
  lineskip=-0.5ex
}


%-----------------------------------------------------------------------------------------
% ページレイアウトは 1行45文字 x 40行
%123456789012345678901234567890123456789012345
%-----------------------------------------------------------------------------------------

%=========================================================================================
% 以下の (1) 課題番号 (2) 氏名(学籍番号)(3) 提出年月日 を修正するのを忘れずに
%=========================================================================================

\title{パターン認識 Quiz X-Y}       % X-Y は課題番号なので変更する
\author{荒井秀一 \\ (XX21ZZZ)}      % 氏名 (学籍番号)も変更する
\date{20YY年MM月DD日 提出}          % 提出年月日を変更する

\begin{document}
\maketitle

%=========================================================================================
% この行以下はご自由に
%=========================================================================================

%=========================================================================================
\section{はじめに}

「パターン認識」のQuiz解答用のフォーマットです.
\begin{enumerate}
\item 課題で使いそうな式の書き方や,
\item 課題で使いそうな図の挿入の仕方や,
\item 課題で使いそうな表の書き方や,
\item 課題で使いそうなprogram listの示し方や,
\item 課題で使いそうなgnuplotを用いた図の作成の仕方や,
\item おまけで,実際に課題で使っている式と図も,
\end{enumerate}
全部示しますので,このファイルを利用しつつQuizの解答を作成してください.

%-----------------------------------------------------------------------------------------
\subsection{章立て}
第X節,第Y項のように章立てを指定できます.

\begin{tabular}{rcl}
``節'' & : & \bf \textbackslash  section\{タイトル\} \\
``項'' & : & \bf \textbackslash  subsection\{タイトル\} 
\end{tabular}

のように指定します.

このページの「はじめに」は``節''({\bf \textbackslash section}), 
「章立て」は``項''({\bf \textbackslash subsection})で書いてあります.

%-----------------------------------------------------------------------------------------
\subsection{1行の文字数}
1行の文字数は,
\begin{flushleft}
123456789012345678901234567890123456789012345
\end{flushleft}
のように45文字です.

%-----------------------------------------------------------------------------------------
\subsection{改ページ}
意図的に改ページしたい場合は

{\bf \textbackslash newpage}

と書きます.

%========
\newpage
%========
%=========================================================================================
\section{式の書き方}
%-----------------------------------------------------------------------------------------
数式が美しく描けるというのは\TeX の大きな特徴の1つです.この授業の課題で使いそうな項目を以下に載せますので参考にしてください.

\begin{table}[h]
\begingroup
\renewcommand{\arraystretch}{2.0}
\begin{tabular}{l|l|c}
  \hline
  項目 & 表記 & 表示 \\
  \hline
  \hline
  分数 & \verb|\frac{a}{2}| & $\displaystyle \frac{a}{2}$  \\
  \hline
  積分 & \verb|\int_{-\infty}^{\infty} \exp(-x^2) dx| & $\displaystyle \int_{-\infty}^{\infty} \exp(-x^2) dx $ \\
  \hline
  偏微分 & \verb|\frac{\partial f}{\partial x}| & $\displaystyle \frac{\partial f}{\partial x} $ \\
  \hline
  総和 & \verb|\sum_{n=1}^{N} s_{n}^{2}| & $\displaystyle \sum_{n=1}^{N} s_{n}^{2} $ \\
  \hline
  太字& \verb|(xab) (\bm{xab}) (\mathbf{xab})| & $(xab) (\bm{xab}) (\mathbf{xab})$ \\
  \hline
  関数& \verb|\sin(\theta) \log(x) \exp(x)| & $\sin(\theta) \log(x) \exp(x)$ \\
  \hline
  括弧等& \verb|\left( \frac{\sin(x)}{x} \cdot 2 \pi \cos(x) \right)| & $\displaystyle \left( \frac{\sin(x)}{x} \cdot 2 \pi \cos(x) \right) $ \\
  \hline
  文字& \verb|\alpha \beta \gamma \varepsilon \eta \theta \rho \omega| & $\alpha \beta \gamma \varepsilon \eta \theta \rho \omega$ \\
  \hline
\end{tabular}
\endgroup
\end{table}

さらに,この授業のQuizに出てくる数式をpickupしました.レポートを書くときに利用してください.

%------------------- (式) --------------------------
\begin{align*}
\keytop{Quiz5-1} & \\
\hat{\bm{\mu}}    &= \displaystyle{\frac{1}{N}\sum^N_{k=1}\mathbf{x}_k} \\
\hat{\bm{\Sigma}} &= \displaystyle{\frac{1}{N}\sum^N_{k=1}}(\mathbf{x}_k-\hat{\bm{\mu}})(\mathbf{x}_k-\hat{\bm{\mu}})^t
\end{align*}
%---------------------------------------------------
%------------------- (式) --------------------------
\begin{align*}
\keytop{Quiz5-2} & \\
    &
    \begin{array}{cccrrrrrc}
        \mathbf{x} &=& ( & -0.5, & -0.2, & -0.4, & -1.0, & 0.1 & )
    \end{array}
\end{align*}
%---------------------------------------------------
%------------------- (式) --------------------------
\begin{align*}
\keytop{Quiz5-2} & \\
    p(x_k|\bm{\theta}) &= \displaystyle{\frac{1}{\sqrt{2\pi}\sigma}} \exp \left\{ - \frac{(x_k-\mu)^2}{2\sigma^2}\right\} \\
    p(D|\bm{\theta})   &= \prod_{k=1}^{N} p(x_k|\bm{\theta}) \\
    l(\bm{\theta})     &= \sum_{k=1}^{N} \ln p(x_k|\bm{\theta}) \\
    l(\mu)             &= -\pi \sum_{k=1}^{N} (x_k - \mu)^2 
\end{align*}
%---------------------------------------------------
%------------------- (式) --------------------------
\begin{align*}
\keytop{Quiz6-1} & \\
    \hat{\lambda}_k &= \frac{N_k + \alpha_k -1}{\sum_{m=1}^K (N_m + \alpha_m - 1)}
\end{align*}
%---------------------------------------------------

%------------------- (式) --------------------------
\begin{align*}
\keytop{Quiz6-2} & \\
    p(x^{*}\!=\!k|D) &= \kappa(x^{*},\tilde{\alpha}_{1\cdots K}) = \frac{N_k + \alpha_k}{\sum_{m=1}^K (N_m + \alpha_m )} \\
    \Gamma[t+1] &= t \Gamma[t] 
\end{align*}
%---------------------------------------------------
%------------------- (式) --------------------------
\begin{align*}
\keytop{Quiz6-3} &\\
    &
    \begin{cases}
        \hat{\mu}       &= \displaystyle{\frac{1}{N+\gamma} \left( \sum_{n=1}^{N} x_n + \gamma\delta \right)}\\
        \hat{\sigma}^2  &= \displaystyle \frac{1}{N+2\alpha+3} \left(
                            \sum_{n=1}^{N} (x_n - \hat{\mu})^2 + 2\beta + \gamma(\delta-\hat{\mu})^2 \right)
    \end{cases}
\end{align*}
%---------------------------------------------------
%------------------- (式) --------------------------
\begin{align*}
\keytop{Quiz6-4} &\\
    p(x^{*}|D)
        &= \kappa(\tilde{\alpha},\tilde{\beta},\tilde{\gamma},\tilde{\delta},D) \nonumber \\
        &= \frac{1}{\sqrt{2\pi}} \frac{ \sqrt{\tilde{\gamma}} \tilde{\beta}^{\tilde{\alpha}} }
                                      { \sqrt{\breve{\gamma}} \breve{\beta}^{\breve{\alpha}} }
                                 \frac{\Gamma[\breve{\alpha}]}{\Gamma[\tilde{\alpha}]}
           \\
        &  \mbox{ただし} \nonumber \\
        &  \begin{cases}
                \breve{\alpha} = \displaystyle \tilde{\alpha} + \frac{1}{2} \nonumber \\
                \breve{\beta}  = \displaystyle \frac{{x^{*}}^2}{2} + \tilde{\beta} + \frac{\tilde{\gamma} \tilde{\delta}^2}{2}
                                - \frac{(\tilde{\gamma}\tilde{\delta} + x^{*})^2}{2(\tilde{\gamma}+1)} \nonumber \\
                \breve{\gamma} = \displaystyle \tilde{\gamma} + 1 \nonumber \\
                \breve{\delta} = \displaystyle \frac{\tilde{\gamma}\tilde{\delta} + x^{*}}{\tilde{\gamma} + 1}
            \end{cases}
\end{align*}
%---------------------------------------------------
%------------------- (式) --------------------------
\begin{align*}
\keytop{Quiz6-5} &\\
    p(x^{*}=0|D) &= \frac{N_0+\alpha}{N+\alpha+\beta} 
\end{align*}
%---------------------------------------------------
%------------------- (式) --------------------------
\begin{align*}
\keytop{Quiz6-6} &\\
    \kappa  &=  \frac{1}{\pi^{Nd/2}}
                \frac{|\bm{\Psi}|^{\alpha/2}}{\tilde{|\bm{\Psi}}|^{\tilde{\alpha}/2}}
                \frac{\Gamma_d[\tilde{\alpha}/2]}{\Gamma_d[\alpha/2]}
                \frac{\gamma^{d/2}}{\tilde{\gamma}^{d/2}} \\
            &   \mbox{ただし} \nonumber \\
            &   \begin{cases}
                    \tilde{\alpha}      &= \alpha + N \\
                    \tilde{\bm{\Psi}}   &= \displaystyle
                                           \bm{\Psi} \!+\! \gamma \bm{\delta} \bm{\delta}^t
                                         \!+\! \sum_{n=1}^{N} \mathbf{x}_n \mathbf{x}_n^t
                                         \!-\! \frac{1}{\gamma \!+\! N} \!
                                               \left(\! \gamma \bm{\delta} \!+\! \sum_{n=1}^{N} \mathbf{x}_n \!\right) \! \!
                                               \left(\! \gamma \bm{\delta} \!+\! \sum_{n=1}^{N} \mathbf{x}_n \!\right)^t \\[5pt]
                    \tilde{\gamma}      &= \gamma + N \\[5pt]
                    \tilde{\bm{\delta}} &= \frac{\displaystyle \gamma \bm{\delta} + \sum_{n=1}^{N} \mathbf{x}_n}
                                                {\displaystyle \gamma + N}
                \end{cases} \notag \\
    \mathrm{Tr}[\mathbf{z}\mathbf{z}^t \mathbf{A}^{-1}] &= \mathbf{z}^t \mathbf{A}^{-1} \mathbf{z} 
\end{align*}
%---------------------------------------------------
%------------------- (式) --------------------------
\begin{align*}
\keytop{Quiz7-1} &\\
\bm{\alpha}^t(\mathbf{x} - \mathbf{x}_0) &= 0 \\
    \bm{\alpha}  &= \bm{\mu}_i - \bm{\mu}_j \hspace{10mm}法線ベクトル\\ 
    \mathbf{x}_0 &= \displaystyle{\underbrace{\strut \frac{1}{2}(\bm{\mu}_i + \bm{\mu}_j)}_{中点}
                  - \underbrace{\strut
                        \frac{\sigma^2}{\lVert \bm{\mu}_i - \bm{\mu}_j \rVert^2}
                        \ln\frac{P(\omega_i)}{P(\omega_j)}
                        (\bm{\mu}_i - \bm{\mu}_j)
                    }_{バイアス(切片の役割)}}
\end{align*}
%---------------------------------------------------

\begin{minipage}{0.5\hsize}
%------------------- (式) --------------------------
\begin{align*}
\keytop{Quiz7-3} &\\
    \{\mathbf{x}_n^{(\omega_1)}\} & =    \left\{
                                        \begin{bmatrix} 2 \\ 6 \end{bmatrix},
                                        \begin{bmatrix} 3 \\ 4 \end{bmatrix},
                                        \begin{bmatrix} 3 \\ 8 \end{bmatrix},
                                        \begin{bmatrix} 4 \\ 6 \end{bmatrix}
                                    \right\} \\
    \{\mathbf{x}_n^{(\omega_2)}\} & =   \left\{  
                                        \begin{bmatrix} 1 \\ -2 \end{bmatrix},
                                        \begin{bmatrix} 3 \\ -4 \end{bmatrix},
                                        \begin{bmatrix} 3 \\  0 \end{bmatrix},
                                        \begin{bmatrix} 5 \\ -2 \end{bmatrix}
                                     \right\} 
\end{align*}
%---------------------------------------------------
\end{minipage}
\begin{minipage}{0.5\hsize}
%------------------- (式) --------------------------
\begin{align*}
\keytop{Quiz7-3} &\\
&
\begin{cases}
    \bm{\mu}_1 = \begin{bmatrix} 3 \\ 6 \end{bmatrix}, &
    \bm{\Sigma}_1 = \begin{bmatrix} \displaystyle\frac{1}{2}&0 \\ 0&2 \end{bmatrix}\\
    \bm{\mu}_2 = \begin{bmatrix} 3 \\ -2 \end{bmatrix},&
    \bm{\Sigma}_2 = \begin{bmatrix} 2&0 \\ 0&2 \end{bmatrix}
\end{cases} \\
&
\begin{cases}
\bm{\Sigma}_1^{-1} &= \begin{bmatrix} 2&0 \\ 0&\displaystyle\frac{1}{2} \end{bmatrix} \\
\bm{\Sigma}_2^{-1} &= \begin{bmatrix} \displaystyle\frac{1}{2}&0 \\ 0&\displaystyle\frac{1}{2} \end{bmatrix}
\end{cases}
\end{align*}
%---------------------------------------------------
\end{minipage}

%------------------- (式) --------------------------
\begin{align*}
\keytop{Quiz8-2} &\\
    \delta^{(L)}_k  &= 2 (g^{(L)}_k - b_{kp})\cdot g^{(L)}_k (1 - g^{(L)}_k) \\
    \delta^{(L)}_k  &= 2 (g^{(L)}_k - b_{kp})(1 - (g^{(L)}_k)^2 )
\end{align*}
%---------------------------------------------------
%------------------- (式) --------------------------
\begin{align*}
\keytop{Quiz9-1} &\\
    \{\mathbf{x}_1, t_1 \} &= \left\{\begin{bmatrix} -1 \\ -1 \end{bmatrix}, -1 \right\} \\
    \{\mathbf{x}_2, t_2 \} &= \left\{\begin{bmatrix}  1 \\  1 \end{bmatrix},  1 \right\} \\
    \{\mathbf{x}_3, t_3 \} &= \left\{\begin{bmatrix}  2 \\  0 \end{bmatrix},  1 \right\}
\end{align*}
%---------------------------------------------------
%------------------- (式) --------------------------
\begin{align*}
\keytop{Quiz9-2} &\\
        \{\mathbf{x}_1, t_1 \} &= \{[ 0\  1]^t, 1 \} \\
        \{\mathbf{x}_2, t_2 \} &= \{[ 1\  0]^t, 1 \} \\
        \{\mathbf{x}_3, t_3 \} &= \{[ 0\ -1]^t, 1 \} \\
        \{\mathbf{x}_4, t_4 \} &= \{[ 0\  0]^t,-1 \} \\
        k(\mathbf{x}_i, \mathbf{x}_j) &= (\mathbf{x}_i^{t} \mathbf{x}_j)^2 
\end{align*}
%---------------------------------------------------
%------------------- (式) --------------------------
\begin{align*}
\keytop{Quiz10-1} &\\
        \mbox{目利き結果$X=$和} & \\
        R(\alpha_1|\mbox{和}) =& \lambda_{11}\cdot P(X=\mbox{和}|\omega_1) \cdot P(\omega_1) + \lambda_{12} \cdot P(X=\mbox{和}|\omega_2) \cdot P(\omega_2) \\
        R(\alpha_1|\mbox{和}) =& \lambda_{21}\cdot P(X=\mbox{和}|\omega_1) \cdot P(\omega_1) + \lambda_{22} \cdot P(X=\mbox{和}|\omega_2) \cdot P(\omega_2) \\
        \mbox{目利き結果$X=$輸} & \\
        R(\alpha_1|\mbox{輸}) =& \lambda_{11}\cdot P(X=\mbox{輸}|\omega_1) \cdot P(\omega_1) + \lambda_{12} \cdot P(X=\mbox{輸}|\omega_2) \cdot P(\omega_2) \\
        R(\alpha_1|\mbox{輸}) =& \lambda_{21}\cdot P(X=\mbox{輸}|\omega_1) \cdot P(\omega_1) + \lambda_{22} \cdot P(X=\mbox{輸}|\omega_2) \cdot P(\omega_2)
\end{align*}
%---------------------------------------------------





%========
\newpage
%========
%=========================================================================================
\section{表の書き方}
表は{\bf tabular}環境を使って書きます.
例えば,

\begin{shadebox}
{\small
\begin{verbatim}
    %==== [表] =============================
    \begin{table}
        \begin{center}
            \caption{表のタイトルを書きます}
            % 文書中からの参照のためのラベル
            \label{tbl:例題の表}
            \begin{tabular}{r|c|r}
            \hline \\
            月日 & 県名 & 感染者数 
            \hline \\
            4/1 & 東京都 & 484 \\
            4/15& 東京都 & 2254 \\
            5/1 & 東京都 & 4088 
            \hline \\
            4/1 & 神奈川県 & 87 \\
            4/15 & 神奈川県 & 525 \\
            5/1 & 神奈川県 & 828
            \hline 
            \end{tabular}
        \end{center}
    \end{table}
    %=======================================
    表示された図を参照するには,表\ref{tbl:例題の表}のようにします.
\end{verbatim}
}
\end{shadebox}
のように書けば,以下のように表示されます.
%==== [表] =============================
\begin{table}[h]
    \begin{shadebox}
        \begin{center}
            \caption{表のタイトルを書きます}
            \label{tbl:例題の表}
            \begin{tabular}{r|c|r}
            \hline 
            月日 & 県名 & 感染者数 \\
            \hline 
            4/1 & 東京都 & 484 \\
            4/15& 東京都 & 2254 \\
            5/1 & 東京都 & 4088 \\
            \hline 
            4/1 & 神奈川県 & 87 \\
            4/15 & 神奈川県 & 525 \\
            5/1 & 神奈川県 & 828 \\
            \hline 
            \end{tabular}
        \end{center}
        \vspace{5mm}
        \mbox{表示された図を参照するには,表\ref{tbl:例題の表}のようにします.}
    \end{shadebox}
\end{table}
%=======================================

%========
\newpage
%========
%=========================================================================================
\section{図の書き方}
図は,{\bf inkscape}などで作成します.
{\bf inkscape}は建築やデザイン業界でよく用いられる{\bf svg} formatのファイルで図を保存します.

\TeX に取り込む図は,{\bf eps} formatのファイルが,大きさを変更しても品質劣化が無いので便利です.

{\bf inkscape}で作成した図は,{\bf eps}フォーマットの図のコピーを生成できます..

{\bf jpg, png}などの``{\bf 画像}用''のformatは``{\bf 図}用''には{\bf 使いません}ので注意してください.

作成した図のファイル(例えば{\bf figure1.eps})を \TeX に取り込むのは,次のようにします.

\begin{shadebox}
{\small
\begin{verbatim}
    %==== [図] =============================
    \begin{figure}
        \begin{center}
        % 幅は用紙表示幅の50パーセント
        \includegraphics[width=0.5\hsize]{./Fig/sample_figure1.eps}
        % 図の下に図番号とタイトルが表示されます
        \caption{図のタイトルを書きます}
        % 文書中からの参照のためのラベル
        \label{fig:例題の図}
        \end{center}
    \end{figure}
    %=======================================
    表示された図を参照するには,図\ref{fig:例題の図}のようにします.
\end{verbatim}
}
\end{shadebox}

実際の表示例は以下のようになります.
%==== [図] =============================
\begin{figure}[htb]
    \begin{shadebox}
        \begin{center}
        \includegraphics[width=0.5\hsize]{./Fig/sample_figure1.eps} 
        \caption{図のタイトルを書きます}                        
        \label{fig:例題の図}                                   
        \end{center}

        \vspace{5mm}
        \mbox{表示された図を参照するには,図\ref{fig:例題の図}のようにします.}
    \end{shadebox}
\end{figure}
%=======================================

%========
\newpage
%========
では,Quizで登場する図を載せておきますので,利用ください.
当然,原図のsvg formatも提供しますので,必要に応じて図を作り変えてください.

%==== [図] =============================
\begin{figure}[htb]
    \begin{center}
    \includegraphics[width=0.8\hsize]{./Fig/Quiz1-4.eps} 
    \caption{Quiz1-4}                        
    \end{center}
\end{figure}
%=======================================
%==== [図] =============================
\begin{figure}[htb]
    \begin{center}
    \includegraphics[width=0.6\hsize]{./Fig/Quiz2-1.eps} 
    \caption{Quiz2-1}                        
    \end{center}
\end{figure}
%=======================================
%==== [図] =============================
\begin{figure}[htb]
    \begin{center}
    \includegraphics[width=0.5\hsize]{./Fig/Quiz2-2.eps} 
    \caption{Quiz2-2}                        
    \end{center}
\end{figure}
%=======================================
%==== [図] =============================
\begin{figure}[htb]
    \begin{center}
    \includegraphics[width=0.4\hsize]{./Fig/Quiz7-3.eps} 
    \caption{Quiz7-3}                        
    \end{center}
\end{figure}
%=======================================

%==== [図] =============================
\begin{figure}[htb]
    \begin{center}
    \includegraphics[width=0.3\hsize]{./Fig/Quiz9-1.eps} 
    \caption{Quiz9-1}                        
    \end{center}
\end{figure}
%=======================================
%==== [図] =============================
\begin{figure}[htb]
    \begin{center}
    \includegraphics[width=0.3\hsize]{./Fig/Quiz9-2.eps} 
    \caption{Quiz9-2}                        
    \end{center}
\end{figure}
%=======================================
%==== [図] =============================
\begin{figure}[htb]
    \begin{center}
    \includegraphics[width=0.3\hsize]{./Fig/Quiz9-3.eps} 
    \caption{Quiz9-3}                        
    \end{center}
\end{figure}
%=======================================
%========
\clearpage
%========
もう1つ,グラフと言えば,{\bf gnuplot}を使ったグラフがあります.
これもサンプル({\bf Quiz9-1.plt})を{\bf ./Plot/}に置きます.

%--------------------------------------------------------
\begin{lstlisting}[caption='Quiz9-1用gnuplot file', label=src:gnuplot]
%--------------------------------------------------------
#!/usr/bin/gnuplot -persist
#
# Quiz9-1 用のplot
#
set grid            # 目盛り罫線
set nokey           # 凡例なし
set size square     # 縦横比率1:1

# データ
set label 1 point pt 7 ps 1.5 lc rgbcolor 'red' at -1,-1
set label 2 point pt 7 ps 1.5 lc rgbcolor 'blue' at 1,1
set label 3 point pt 7 ps 1.5 lc rgbcolor 'blue' at 2,0

set xrange [-3:3]   # X軸 表示範囲
set yrange [-3:3]   # Y軸 表示範囲
set xlabel 'x'
set ylabel 'y'

# 関数 (y=f(x)の形で書ける場合)
f(x) = -2.0 * x

# 画面表示
set term 'x11'
plot f(x) lc rgbcolor 'magenta' lw 2

# ファイル保存
set term 'svg'
set output 'quiz9-1.1.svg'
plot f(x) lc rgbcolor 'magenta' lw 2

pause -1 'Hit return key'

# 関数(y=f(x)の形では書けない場合は媒介変数tを使った表示で)
set parametric
g_x(t) = cos(t)
g_y(t) = sin(t)

# 画面表示
set term 'x11'
plot g_x(t), g_y(t) lc rgbcolor 'magenta' lw 2

# ファイル保存
set term 'svg'
set output 'quiz9-1.2.svg'
plot g_x(t), g_y(t) lc rgbcolor 'magenta' lw 2
%--------------------------------------------------------
\end{lstlisting}
%--------------------------------------------------------
このファイルを使って,{\bf gnuplot Quiz9-1.plt} と実行すると,2つのsvg formatのファイルが生成されます.
これをinkscapeで読み込んで,必要な修正や追加を行い,eps formatで保存します.

例えばinkscapeで{\bf quiz9-1.2.svg}というファイルを開いて,「ファイル」メニューで「コピーを保存」を選ぶと色々なフォーマットでコピー保存ができます.
その中に{\bf "EPS - Encapsulated PostScript (*.eps)} がありますので,\TeX で取り込むためには,このフォーマットで保存してください.

それを表示すると,以下のようになります.

\begin{figure}[htb]
\begin{minipage}{0.5\hsize}
%==== [図] =============================
    \begin{center}
    \includegraphics[width=\hsize]{./Plot/quiz9-1.1.eps} 
    \caption{Quiz9-1.1}                        
    \end{center}
%=======================================
\end{minipage}
\begin{minipage}{0.5\hsize}
%==== [図] =============================
    \begin{center}
    \includegraphics[width=\hsize]{./Plot/quiz9-1.2.eps} 
    \caption{Quiz9-1.2}                        
    \end{center}
%=======================================
\end{minipage}
\end{figure}

%========
\newpage
%========
%=========================================================================================
\section{Program listの書き方}

例えば,有名なHelloWorldのソースコードをレポートに書きたいときには,以下のようにします.
\begin{shadebox}
{\small
\begin{verbatim}
    %==== [Prog] ===========================
    \begin{lstlisting}[caption='Hello World', label=src:hello]
    /* C言語での例 */
    #include <stdio.h>
    int main()
    {
        printf("Hello World\n");
    }
    \end{lstlisting}
    %=======================================

    参照するときには, ソースコード\ref{src:hello}}のようにします.
\end{verbatim}
}
\end{shadebox}

すると,以下のように表示されます.

\begin{shadebox}
%==== [Prog] ===========================
\begin{lstlisting}[caption='Hello World', label=src:hello]
/* C言語での例 */
#include <stdio.h>
int main()
{
    printf("Hello World\n");
}
\end{lstlisting}
    参照するときには, ソースコード\ref{src:hello}のようにします.
%=======================================
\end{shadebox}


%=========================================================================================
\end{document}
%=========================================================================================
